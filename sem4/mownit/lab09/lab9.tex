\documentclass{article}
\usepackage[polish]{babel}
\usepackage[T1]{fontenc}
\usepackage{amsmath}
\usepackage{graphicx} % Required for inserting images

\title{Laboratorium 9 \\ Równania różniczkowe zwyczajne - część I}
\author{Maciej Borowiec}
\date{20.05.2025}

\begin{document}

\maketitle

\section{Wprowadzenie}
Celem ćwiczenia było zrobienie sześciu zadań związanych z równaniami różniczkowymi zwyczajnymi. Zadania miały na celu przedstawienie sposobów rozwiązywania równań różniczkowych.

\section{Zadania}

\subsection{Zadanie 1.}
Zadanie polegało na przekształcenie podanych równań różniczkowych na równoważny układ równań pierwszego rzędu.

\subsubsection{Równanie Van der Pol'a}
$$ y'' = y'(1- y^2) - y $$
Podstawiamy $u_1 = y$, $u_2 = y'$, wtedy:
\begin{equation}
    \begin{cases}
        u_1' = u_2 \\
        u_2' = u_2(1-u_1^2)-u_1
    \end{cases} \nonumber
\end{equation}
Otrzymany układ równań jest układem równań pierwszego rzędu.

\subsubsection{Równanie Blasiusa}
$$ y''' = -yy'' $$
Podstawiamy $u_1 = y$, $u_2 = y'$, $u_3 = y''$, wtedy:
\begin{equation}
    \begin{cases}
        u_1' = u_2 \\
        u_2' = u_3 \\
        u_3' = -u_1u_3
    \end{cases} \nonumber
\end{equation}
Ponownie otrzymano układ równań pierwszego rzędu.

\subsubsection{II zasada dynamiki Newtona dla problemu dwóch ciał}
\begin{equation}
    \begin{cases}
        y_1'' = -GMy_1/(y_1^2+y_2^2)^{3/2} \\
        y_2'' = -GMy_2/(y_1^2+y_2^2)^{3/2}
    \end{cases} \nonumber
\end{equation}
Podstawiamy $u_1 = y_1$, $u_2 = y_2$, $u_3 = y_1'$, $u_4 = y_2'$, wtedy:
\begin{equation}
    \begin{cases}
        u_1' = u_3 \\
        u_2' = u_4 \\
        u_3' = -GMu_1/(u_1^2+u_2^2)^{3/2} \\
        u_4' = -GMu_2/(u_1^2+u_2^2)^{3/2}
    \end{cases} \nonumber
\end{equation}
Otrzymany układ równań jest układem równań pierwszego rzędu.

\subsection{Zadanie 2.}
W tym zadaniu należało przekształcić poniższy problem początkowy do autonomicznego problemu początkowego.
\begin{equation}
    \begin{cases}
        y_1' = \frac{y_1}{t} + y_2t \\
        y_2' = \frac{t}{y_1}(y_2^2 - 1) \\
    \end{cases} \nonumber
    \begin{cases}
        y_1(1) = 1 \\
        y_2(1) = 0 \\
    \end{cases} \nonumber
\end{equation}
\\
Aby poprawnie przekształcić problem do problemu początkowego, należy dodać nową zmienną tożsamościowo równą $t$. W tym przypadku dodajemy zmienną $y_3(t) = t$. Wtedy otrzymujemy:
\begin{equation}
    \begin{cases}
        y_1' = \frac{y_1}{y_3} + y_2y_3 \\
        y_2' = \frac{y_3}{y_1}(y_2^2 - 1) \\
        y_3' = 1
    \end{cases} \nonumber
    \begin{cases}
        y_1(1) = 1 \\
        y_2(1) = 0 \\
        y_3(1) = 1
    \end{cases} \nonumber
\end{equation}
W ten sposób otrzymaliśmy autonomiczny problem początkowy.

\subsection{Zadanie 3.}
Dla danego problemu początkowego:
\begin{equation}
    y' = \sqrt{1-y}
\end{equation}
$$ y(0) = 0 $$
Należało pokazać, że funkcja $y(t) = \frac{t}{4}(4-t)$ spełnia równanie i warunek początkowy oraz wyznaczyć dziedzinę, dla której ta funkcja jest rozwiązaniem tego problemu.
\\\\
Sprawdzenie warunku początkowego:
$$ y(0) = \frac{0}{4}(4-0) = 0$$
Następnie wyznaczmy pochdną funkcji $y(t)$:
$$ y'(t) = (\frac{t}{4}(4-t))' = (t - \frac{t^2}{4})' = 1 - \frac{t}{2} $$
Podstawmy do problemu początkowego (1):
$$ 1-\frac{t}{2} = \sqrt{1 - t + \frac{t^2}{4}} = \sqrt{(1-\frac{t}{2})^2}$$
$$ 1-\frac{t}{2} = |1-\frac{t}{2}|$$
Zatem dana funkcja jest rozwiązaniem tego problemu wtw., gdy $1-\frac{t}{2} \geq 0$.
$$ 1-\frac{t}{2} \geq 0 $$
$$ t \leq 2$$
Zatem dziedziną tego rozwiązania jest $D = (-\infty;2]$.

\subsection{Zadanie 4.}

\subsection{Zadanie 5.}
Dla danego układu równań różniczkowych zwyczajnych:
\begin{equation}
    \begin{cases}
        y_1' = -2y_1 + y_2 \\
        y_2' = -y_1 - 2y_2
    \end{cases} \nonumber
\end{equation}
Należało wyznaczyć przedział wartości kroku $h$, dla którego metoda Euler'a jest stablina dla tegu układu równań.
\\\\
Układ równań można przedstawić w postaci macierzowej:
$$ y' = Ay$$
gdzie:
\begin{equation}
    y' =
    \begin{bmatrix}
        y_1' \\ y_2'
    \end{bmatrix}
    ,\quad
    A =
    \begin{bmatrix}
        -2 & 1 \\
        1 & -2
    \end{bmatrix}
    ,\quad
    y =
    \begin{bmatrix}
        y_1 \\ y_2
    \end{bmatrix}
\nonumber
\end{equation}
Wtedy wiadomo, że metoda Eulera jest stabilna, gdy:
$$ 0 < h < \frac{2}{\lambda_{max}} $$
$$ \lambda_{max} = \max \{|\lambda|\ \ |\ \ \lambda\text{ to wartość własna macierzy A}\}$$
Wartości własne macierzy możemy wyznaczyć następująco:
$$ \det(A-\lambda I) = 0 $$
\begin{equation}
    \begin{vmatrix}
        -2-\lambda & 1 \\
        -1 & -2-\lambda
    \end{vmatrix}
    = 0
\nonumber
\end{equation}
Wtedy:
$$ (-2-\lambda)^2 + 1 = \lambda^2 + 4\lambda + 5 = 0 $$
$$ \Delta = 4^2 - 4*5 = -4, \quad \sqrt{\Delta} = \pm2i $$
$$ \lambda_1 = \frac{-4-2i}{2} = -2-i \quad \lambda_2 = \frac{-4+2i}{2} = -2+i $$
$$ |\lambda_1| = \sqrt{5} \quad |\lambda_2| = \sqrt{5} $$
$$ \lambda_{max} = \sqrt{5}$$
Zatem metoda Eulera jest stabilna dla tego układu równań, gdy:
$$ 0 < h < \frac{2}{\sqrt{5}} = \frac{2\sqrt{5}}{5}$$

\subsection{Zadanie 6.}

\section{Podsumowanie}

\section{Bibliografia}

\end{document}
